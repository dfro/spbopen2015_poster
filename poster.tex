%TODO нужно доавить про описание образцов +  сравнение с GaAs
%TODO задача отделить влияние паразитов от свойств реальной структуры.
%TODO незабыть обозначить, что иследовался n-InAs
%\documentclass[final,professionalfont]{beamer}
\documentclass[final]{beamer}
\mode<presentation>

% STEP 1: Change the next line according to your language
\usepackage[english]{babel}

% STEP 2: Make sure this character encoding matches the one you save this file as
% (this template is utf8 by default but your editor may change it, causing problems)
\usepackage[utf8x]{inputenc}

% You probably don't need to touch the following four lines
\usepackage[T1]{fontenc}
\usepackage{lmodern}
\usepackage{amsmath,amsthm, amssymb, latexsym}
\usepackage{calc}
\usepackage{exscale} % required to scale math fonts properly
%\usepackage{eulervm} % Euler VM symbols
%\usepackage[cmbrightmath,scaleupmath]{tpslifonts}
%\usepackage{cmbright} % CM Bright math
\usepackage{ragged2e} 
\usepackage{svg}
\graphicspath{{images/}}
%\usepackage[orientation=portrait,size=a0,scale=1.3]{beamerposter}
\usepackage[orientation=portrait,size=custom,width=65,height=100,scale=0.95]{beamerposter}

% STEP 3:
% Change colours by setting \usetheme[<id>, twocolumn]{LETI}.
\usetheme[color, twocolumn]{LETI}



% STEP 4: Set up the title and author info
\titlestart{The measurements of doping density in InAs by
            capacitance-voltage techniques with electrolyte
            barriers} % first line of title
\titleend{} % second line of title
%\titlesize{\Huge} % Use this to change title size if necessary

\author{Dmitry Frolov, Georgy Yakovlev and Vasily Zubkov}
\institute{St. Petersburg Electrotechnical University "LETI", Russia}


% Stuff such as logos of contributing institutes can be put in the lower left corner using this
\leftcorner{}

\newcommand{\figfont}{\normalsize} % set fotsize for figures 

\begin{document}
\begin{poster}
%%%%%%%%%%%%%%%%%%%%%%%%%%%%%%%%%%%%%%%%%%%%%%%%%%%%%%%%%%%%%%%%%%%%%%%%%%%%%%
% First column %%%%%%%%%%%%%%%%%%%%%%%%%%%%%%%%%%%%%%%%%%%%%%%%%%%%%%%%%%%%%%%
\newcolumn

\section{Motivation}
\justifying

%Electrolyte barriers is used in CV measurements of InAs because formation of reliable Schottky contact is difficult due to surface accumulation.
%Although this technique is widely used for characterisation of grate range of semiconductors, the electrolyte-based capacitance-voltage plots of InAs is differ form many other materials and use of depletion approximation give higher concentration compering to hall measurements.
%The purpose of present work is to explain the mismatch between CV and Hall results in undoped samples and to determine the optimal parameters of CV measurements for doping density extraction in n-InAs.

\begin{itemize} 
    \justifying
%    \item Schottky contact is usually used in capacitance-voltage characterisation         
    \item Formation of reliable Schottky contact in InAs is difficult due to surface accumulation
    \item Electrolyte can be used to form Schottky-like barrier
    \item Depletion approximation for electrolyte based capacitance-voltage characteristics gives higher concentration compering to Hall measurements
\end{itemize}

\section{Semiconductor-electrolyte interface} \justifying

\centering{
\figfont
\includesvg[clean, width=0.7\columnwidth]{images/sem_el}
\includesvg[clean, width=0.45\columnwidth]{images/sem_el_cap}
\vspace{1em}
\caption{The structure of semiconductor-electrolyte interface}  
}

\vspace{3ex}
\begin{columns}
    \figfont
    \newcommand{\figwidth}{0.5555\columnwidth}
    \begin{column}{\figwidth}
        \centering{
        \includesvg[clean, width=0.8\columnwidth]{images/IP_band_edge1}
        \caption{Position of energy bands at the surface of various semiconductors in aqueous solution}
        }
    \end{column}
    \begin{column}{\columnwidth-\figwidth}
        \includesvg[clean, width=0.8\columnwidth]{images/IP_band_edge2}
        \centering{
        \caption{Band diagrams of InAs add GaAs at equilibrium}
        }
    \end{column}
\end{columns}
\vspace{-2ex}
\section{Simulation} \justifying
%Capacitance-voltage characteristics were calculated from potential distributions obtained by solving Poisson equation with modified Thomas-Fermi approximation (MTFA).   

Poisson equation with modified Thomas-Fermi approximation (MTFA) was used to calculate CV characteristics.

\subsection{Poisson equation}

    $$
        \frac{d^2\varphi}{dz^2} =
         -\frac{q}{\varepsilon\varepsilon_0}\left[N_D^+ - N_A^- - n(z) + \
          p(z)\right] 
    $$
    
    with electron concentration:
    $$
        n(z) = \int_{0}^{\infty}\rho_c(z,E) f_{FD}(E) f_{MTFA}(z,E)dE
    $$ 
    
    and  DOS for non-parabolic conduction band:  
    $$
        \rho \left(z,E\right) = \frac{1}{2\pi^2} \left(\frac{2m_{\Gamma}}{\hbar^2}\right)^{3/2} \!\!\! \sqrt{E} \cdot \sqrt{1+\alpha E} \cdot \left(1+ 2\alpha E \right)
    $$  
    
%    $$
%   \text{where}\ \alpha = \frac{1}{E_g}\left(1 - \frac{m_{\Gamma}}{m_0}\right)^{\!2}\ \ \text{--- nonparabolicity coefficient}
%    $$

       
\subsection{Modified Thomas-Fermi approximation}
    MTFA used to take into account boundary condition for wave function during accumulation.  
    $$
    f_{MTFA}(z, E)  = 1 - sinc\left( \frac{2z}{L} \left(\frac{E}{k_BT}\right)^{1/2} \left(1+\alpha E\right)^{1/2}\right)
    $$
    
%%%%%%%%%%%%%%%%%%%%%%%%%%%%%%%%%%%%%%%%%%%%%%%%%%%%%%%%%%%%%%%%%%%%%%%%%%%%%%
% Second column %%%%%%%%%%%%%%%%%%%%%%%%%%%%%%%%%%%%%%%%%%%%%%%%%%%%%%%%%%%%%%
\newcolumn

\section{Results} \justifying
\subsection{Energy band diagrams}
        \centering{
        \figfont
        \includesvg[clean, width=\columnwidth]{images/IP_InAs_pot}
        }
At different biases ...

\subsection{Standard Mott-Schottky plot}
\newcommand{\figwidth}{0.55\columnwidth}
\begin{columns}[c]
    \begin{column}{0.95\columnwidth-\figwidth}

    \end{column}
    
    \begin{column}{0.005\columnwidth}
    \end{column}
    
    \begin{column}{\figwidth}
        \centering
        \figfont
        \includesvg[clean, width=\columnwidth]{images/IP_InAs_CV_fig3}
        \caption{Mott-Schotky plot of n-GaAs}  
    \end{column}
\end{columns}	

\subsection{Mott-Schottky plots of InAs-electrolyte}
%\newcommand{\figwidth}{0.55\columnwidth}
\begin{columns}[c]
        \begin{column}{0.95\columnwidth-\figwidth}
                \begin{itemize}   \itemsep20pt       
                    \item $N_D = 2\times10^{18}\ \mathsf{cm^{-3}}$ from Hall measurements. Same value used in simulation.
                    \item Capacitance measurements frequency: $f_m=2\ \mathsf{kHz}$.
                    \item Accumulation at low positive slope.
                    \item Depletion at linear region.
                \end{itemize}
        \end{column}
        
        \begin{column}{0.005\columnwidth}
        \end{column}
        
        \begin{column}{\figwidth}
           \centering
           \figfont
           \includesvg[clean, width=\columnwidth]{images/IP_InAs_CV_fig1}
           \caption{Mott-Schotky plot of $n^+\!$-InAs}  
        \end{column}
\end{columns}
   
\begin{columns}[c]
        \begin{column}{0.95\columnwidth-\figwidth}
            \begin{itemize}    \itemsep20pt          
                \item $N_D = 1\times10^{15}\ \mathsf{cm^{-3}}$.
                \item $f_m=1\ \mathsf{MHz}$.
                \item Depletion region is very small.
                \item Fermi level shifts toward the
                center of forbidden energy gap, and the inversion starts earlier.
               \end{itemize}
        \end{column}
        
        \begin{column}{0.005\columnwidth}
        \end{column}   
                   
        \begin{column}{\figwidth}
             \centering
             \figfont
             \includesvg[clean, width=\columnwidth]{images/IP_InAs_CV_fig2}
             \caption{Mott-Schotky plot of epi-InAs}  
        \end{column}
\end{columns}	

\section{Summary} \justifying
    \begin{itemize} \itemsep12pt
        \justifying
        \item     In heavily doped n-InAs ($\mathsf{N_D > 10^{18} cm^{-3}}$) the range of biases, in which the depletion occurs, is wide enough to use the depletion approximation for doping density estimation with 10  \% accuracy.
        \item At the lower doping levels simulation of capacitance-voltage characteristics should be used         to estimate doping density.
    \end{itemize}


\end{poster}
\end{document}